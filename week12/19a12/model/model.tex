\documentclass[acmlarge,screen=true,anonymous=false,11pt]{acmart}

% ======================================================== 
%          Copyright & Conference setting
% ======================================================== 
% copyright
\setcopyright{none}
\settopmatter{printacmref=false, printccs=false, printfolios=true}

%\editor{Jennifer B. Sartor}
%\editor{Theo D'Hondt}
%\editor{Wolfgang De Meuter}

\usepackage{afterpage}
\usepackage{pdflscape}
\usepackage{pdfpages}

\usepackage{cleveref}
\crefname{mytheorem}{Theorem}{Theorems}

\usepackage{listings}

\usepackage{pgfplots}
\usepackage{textgreek}
\usepgfplotslibrary{groupplots}
\usetikzlibrary{patterns}

\usepackage{version}
% \excludeversion{proof}
\newtheorem*{solution*}{Solution}
% \excludeversion{solution*}

% ==== defined colors
\definecolor{myorange}{RGB}{244,106,18} %F47012
\definecolor{myblue}{RGB}{0,111,190}    %006FBE
\definecolor{mygreen}{RGB}{0,127,128}   %007F80
\definecolor{myred}{RGB}{228,46,36}     %E42E24
\definecolor{myyellow}{RGB}{198,148,34} %C69422
\definecolor{mydark}{RGB}{114,44,114}   %722C72
\definecolor{mymiddle}{RGB}{144,44,144} %902C90
\definecolor{mylight}{RGB}{167,44,167}  %A72CA7

\definecolor{light-cayenne}{RGB}{177,0,28} %B1001C
\definecolor{light-mocha}{RGB}{179,68,0}   %B34400
\definecolor{ocean}{RGB}{0,55,118}         %003776

\iffalse
\newtoggle{screen}
\toggletrue{screen}
\AtEndPreamble{%
\iftoggle{screen}
{
    \hypersetup{colorlinks, % 01/2018 Bei: change link colors all to blue
        linkcolor=blue,
        citecolor=blue,
        urlcolor=blue,
        filecolor=blue}
}
{
    \hypersetup{hidelinks}
}}
\fi

\begin{document}
\title{CENG 4120 Assignment 3}
\subtitle{Computer-Aided Design for Very Large Scale Integrated Circuits \\ Due Date: April 15 (Monday) 5:00pm}

\maketitle

\section{Library Analysis}
\href{https://www.eda.ncsu.edu/wiki/FreePDK45:Contents}{FreePDK 45nm Standard Cell Libaray}
is used in this homework.
Connect to a Department Unix server with
\href{https://download.mobatek.net/1112019010310554/MobaXterm_Portable_v11.1.zip}{MobaXterm}
or any other Secure Shell (SSH) client you are familiar with.
Upload \texttt{19hw3-src.zip} to the Department Unix server.
Extract the file with command \texttt{unzip 19hw3-src.zip}.
Open liberty file \texttt{gscl45nm.tlf} in the directory with a text editor.
\begin{enumerate}
    \item Find cell \texttt{INVX1}.
    \begin{example}
        What is the cell area?
    \end{example}
    \begin{solution*}
        $1.407900~\mathrm{\mu m^2}$.
    \end{solution*}
    \begin{example}
        What is the capacitance of pin \texttt{A}?
    \end{example}
    \begin{solution*}
        $0.001551~\mathrm{pF}$.
    \end{solution*}
    \begin{example}
        What is the function of pin \texttt{Y}?
    \end{example}
    \begin{solution*}
        (!A).
    \end{solution*}
    \item Find cell \texttt{FAX1}.
    \begin{example}
        What is the cell area?
    \end{example}
    \begin{solution*}
        $8.916700~\mathrm{\mu m^2}$.
    \end{solution*}
    \begin{example}
        What is the capacitance of pin \texttt{B}?
    \end{example}
    \begin{solution*}
        $0.010305~\mathrm{pF}$.
    \end{solution*}
    \begin{example}
        What is the function of pin \texttt{YC}?
    \end{example}
    \begin{solution*}
        (((A B)+(B C))+(C A)).
    \end{solution*}
    \item Find cell \texttt{DFFPOSX1}.
    \begin{example}
        What is the cell area?
    \end{example}
    \begin{solution*}
        $7.978100~\mathrm{\mu m^2}$.
    \end{solution*}
    \begin{example}
        What is the cell leakage power?
    \end{example}
    \begin{solution*}
        $54.977402~\mathrm{nW}$
    \end{solution*}
    \begin{example}
        What is the capacitance of pin \texttt{D}?
    \end{example}
    \begin{solution*}
        $0.001816~\mathrm{pF}$.
    \end{solution*}
\end{enumerate}

\section{Physical Design}
\label{sec:pd}
Change to the \texttt{19hw3-src} directory. Run \texttt{source setup.sh} in the directory to setup the environment variables of Cadence Innovus. Initialize a design with synthesized Verilog net-list \texttt{multiplyadd.vh}, library exchange file \texttt{gscl45nm.lef}, MMMC file \texttt{multiplyadd.view}, ground net \texttt{gnd} and power net \texttt{vdd}.
\begin{example}
    Floor-plan the design with aspect ratio 1.0, row density 0.6, core margin $20~\mathrm{\mu m}$ in all four directions. What are the width and height of the placement grid?
\end{example}
\begin{solution*}
    Width is $60.8~\mathrm{\mu m}$. Height is $56.81~\mathrm{\mu m}$.
\end{solution*}
\begin{example}
    Create rings for nets \texttt{gnd} and \texttt{vdd} centering between the I/O pads and core boundaries. Set the top/bottom wires on \texttt{metal3} and the left/right wires on \texttt{metal4}. Set the wire spacing as $5~\mathrm{\mu m}$ and width as $5~\mathrm{\mu m}$. How many wires and vias are created?
\end{example}
\begin{solution*}
    8 wires and 8 vias are created.
\end{solution*}
\begin{example}
    Turn on auto via generation for the \texttt{NanoRoute} router.
    Then place design. How many instances are placed?
\end{example}
\begin{solution*}
    557 instances are placed.
\end{solution*}
\begin{example}
    Set the maximum layer of pin assignment as 4 and assign I/O pins.
    Route power structures. Which layers are the power structures routed?
\end{example}
\begin{solution*}
    The power structures are routed in \texttt{metal1}.
\end{solution*}
\begin{example}
    Insert filler cell instances of \texttt{FILL} in the gaps between standard cell instances. How many filler instances are added?
\end{example}
\begin{solution*}
    2254 filler instances are added.
\end{solution*}
\begin{example}
    Connect the tie high pins to global net \texttt{vdd}.
    Connect the power and ground pins with name \texttt{vdd} to global net \texttt{vdd}
    and override previous setting.
    Connect the tie low pins to global net \texttt{gnd}.
    Connect the power and ground pins with name \texttt{gnd} to global net \texttt{gnd}
    and override previous setting.
    Route the design with the \texttt{NanoRoute} router.
    What is the total wire length?
    How many single-cut vias are used?
    How many multi-cut vias are used?
\end{example}
\begin{solution*}
    The total wire length is $6674~\mathrm{\mu m}$.
    425 single-cut vias and 1998 multi-cut vias are used.
\end{solution*}
\begin{example}
    Extract resistance and capacitance for the interconnections using the post-route engine.
    Generate a timing report.
    What is the slack?
\end{example}
\begin{solution*}
    The slack is $7.447$.
\end{solution*}
\begin{example}
    Attach all scripts you used in file \texttt{multiplyadd.tcl}.
\end{example}
\begin{solution*}
    \lstset{language=tcl,
        basicstyle=\ttfamily\scriptsize
    }
    \begin{lstlisting}
    set top multiplyadd
    set init_verilog ${top}.vh
    set init_top_cell ${top}
    set init_design_netlisttype Verilog
    set init_lef_file gscl45nm.lef
    set init_pwr_net vdd
    set init_gnd_net gnd
    set init_mmmc_file ${top}.view
    init_design
    
    # Create Initial Floorplan
    floorplan -r 1.0 0.6 20 20 20 20
    
    # Create Power structures
    addRing -spacing 5 -width 5 -layer {top metal3 bottom metal3 left metal4 right metal4}\
        -center 1 -nets { gnd vdd }
    
    specifyScanChain ${top}_sc -start a[0] -stop result[16]
    setScanReorderMode -compLogic true
    setNanoRouteMode -routeUseAutoVia true
    setPinAssignMode -maxLayer 4
    assignIoPins
    # Place standard cells
    place_opt_design
    
    # Route power nets
    sroute
    
    # Add filler cells
    addFiller -cell FILL -prefix FILL -fillBoundary
    
    # Connect all new cells to VDD/GND
    globalNetConnect vdd -type tiehi
    globalNetConnect vdd -type pgpin -pin vdd -override
    globalNetConnect gnd -type tielo
    globalNetConnect gnd -type pgpin -pin gnd -override
    # Run global Routing
    routeDesign
    
    # Get final timing results
    setExtractRCMode -engine postRoute
    extractRC
    buildTimingGraph
    setCteReport
    report_timing
    \end{lstlisting}
\end{solution*}
\section{Timing Optimization}
Repeat~\Cref{sec:pd} for different row density 0.3, 0.4, 0.5, 0.6, 0.7, 0.8, 0.9.
\begin{example}
    Show the relationship between row density and the wire length in each layer.
\end{example}
\begin{example}
    Show the relationship between row density and the number of single/multi-cut vias.
\end{example}
\begin{example}
    Show the relationship between row density and the slack.
\end{example}

\end{document}
