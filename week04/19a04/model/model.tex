\documentclass[acmlarge,screen=true,anonymous=false,11pt]{acmart}

% ======================================================== 
%          Copyright & Conference setting
% ======================================================== 
% copyright
\setcopyright{none}
\settopmatter{printacmref=false, printccs=false, printfolios=true}

%\editor{Jennifer B. Sartor}
%\editor{Theo D'Hondt}
%\editor{Wolfgang De Meuter}

\usepackage{afterpage}
\usepackage{pdflscape}
\usepackage{pdfpages}

\usepackage{cleveref}
\crefname{mytheorem}{Theorem}{Theorems}

\usepackage{listings}

\usepackage{pgfplots}
\usepackage{textgreek}
\usepgfplotslibrary{groupplots}
\usetikzlibrary{patterns}

\usepackage{version}
% \excludeversion{proof}
\newtheorem*{solution*}{Solution}
\excludeversion{solution*}

% ==== defined colors
\definecolor{myorange}{RGB}{244,106,18} %F47012
\definecolor{myblue}{RGB}{0,111,190}    %006FBE
\definecolor{mygreen}{RGB}{0,127,128}   %007F80
\definecolor{myred}{RGB}{228,46,36}     %E42E24
\definecolor{myyellow}{RGB}{198,148,34} %C69422
\definecolor{mydark}{RGB}{114,44,114}   %722C72
\definecolor{mymiddle}{RGB}{144,44,144} %902C90
\definecolor{mylight}{RGB}{167,44,167}  %A72CA7

\definecolor{light-cayenne}{RGB}{177,0,28} %B1001C
\definecolor{light-mocha}{RGB}{179,68,0}   %B34400
\definecolor{ocean}{RGB}{0,55,118}         %003776

\iffalse
\newtoggle{screen}
\toggletrue{screen}
\AtEndPreamble{%
\iftoggle{screen}
{
    \hypersetup{colorlinks, % 01/2018 Bei: change link colors all to blue
        linkcolor=blue,
        citecolor=blue,
        urlcolor=blue,
        filecolor=blue}
}
{
    \hypersetup{hidelinks}
}}
\fi

\begin{document}
\title{CENG 4120 Assignment 1}
\subtitle{Computer-Aided Design for Very Large Scale Integrated Circuits \\ Due Date: February 18 (Monday) 5:00pm}

\maketitle

\section{Library Analysis}
\href{https://www.eda.ncsu.edu/wiki/FreePDK45:Contents}{FreePDK 45nm Standard Cell Libaray} is used in this homework.
Open a terminal in a Department machine and upload \texttt{19hw1-src.zip}.
Extract the file with command \texttt{unzip 19hw1-src.zip}.
Open liberty file \texttt{gscl45nm.lib} in the directory with a text editor.
\begin{enumerate}
    \item Find cell \texttt{INVX1}.
    \begin{example}
        What is the cell area?
    \end{example}
    \begin{solution*}
        $1.407900$.
    \end{solution*}
    \begin{example}
        What is the function of pin \texttt{Y}?
    \end{example}
    \begin{solution*}
        (!A).
    \end{solution*}
    \item Find cell \texttt{FAX1}.
    \begin{example}
        What is the cell area?
    \end{example}
    \begin{solution*}
        $8.916700$.
    \end{solution*}
    \begin{example}
        What is the function of pin \texttt{YC}?
    \end{example}
    \begin{solution*}
        (((A B)+(B C))+(C A)).
    \end{solution*}
    \item Find cell \texttt{DFFPOSX1}.
    \begin{example}
        What is the cell area?
    \end{example}
    \begin{solution*}
        $7.978100$
    \end{solution*}
    \begin{example}
        What is the cell leakage power?
    \end{example}
    \begin{solution*}
        $54.9774~\mathrm{nW}$
    \end{solution*}
\end{enumerate}

\section{Synthesis}
Change to the \texttt{19hw1-src} directory. Run \texttt{source setup.sh} in the directory to setup the environment variables of Synopsys Design Compiler. Synthesis 2-Bit Carry Look Ahead Adder \texttt{cla2.v} with library \texttt{gscl45nm.db}.
\begin{example}
    Draw a gate-level schematic of the synthesized net-list.
\end{example}
\begin{example}
    What is the total cell area?
\end{example}
\begin{solution*}
    $17.833401$.
\end{solution*}
\begin{example}
    What is the total power consumption?
\end{example}
\begin{solution*}
    $7411.1~\mathrm{nW}$.
\end{solution*}
\begin{example}
    Attach all scripts you used in file \texttt{cla2.tcl}.
\end{example}
\begin{solution*}
    \lstset{language=tcl,
        basicstyle=\ttfamily\scriptsize
    }
    \begin{lstlisting}
        # Compile Script for Synopsys
        # dc_shell-t -f cla2.tcl
        set top cla2
        set target_library "gscl45nm.db"
        set link_library "gscl45nm.db"
        analyze -f verilog ${top}.v
        elaborate ${top}
        compile_ultra
        set filename [format "%s%s"  ${top} ".vh"]
        write_file -f verilog -output $filename
        redirect timing.rep { report_timing }
        redirect cell.rep { report_cell }
        redirect power.rep { report_power }
        quit
    \end{lstlisting}
\end{solution*}
\section{Timing Optimization}
Design a 4-Bit Full Adder with five parameters using Verilog: a 4-bit output summation \texttt{sum}, a 1-bit output carry-out \texttt{cout}, two 4-bit input operands \texttt{a} and \texttt{b}, a 1-bit input carry-in \texttt{cin}. Synthesis your designed adder.
\begin{example}
    Attach your Verilog in file \texttt{cla4.v}.
\end{example}
\begin{solution*}
    \lstset{language=verilog,
        basicstyle=\ttfamily\scriptsize
    }
    \begin{lstlisting}
        module cla4 (sum, cout, a, b, cin);
        
        output  [3:0]   sum;
        output          cout;
        input   [3:0]   a;
        input   [3:0]   b;
        input           cin;
        
        wire    [2:0]   c;
        wire    [3:0]   g;
        wire    [3:0]   p;
        
        wire            and100_out;
        wire            and110_out;
        wire            and111_out;
        wire            and120_out;
        wire            and121_out;
        wire            and122_out;
        wire            and130_out;
        wire            and131_out;
        wire            and132_out;
        wire            and133_out;
        
        and and00(g[0], a[0], b[0]);
        and and01(g[1], a[1], b[1]);
        and and02(g[2], a[2], b[2]);
        and and03(g[3], a[3], b[3]);
        
        or or00(p[0], a[0], b[0]);
        or or01(p[1], a[1], b[1]);
        or or02(p[2], a[2], b[2]);
        or or03(p[3], a[3], b[3]);
        
        and and100(and100_out, p[0], cin);
        
        and and110(and110_out, p[1], p[0], cin);
        and and111(and111_out, p[1], g[0]);
        
        and and120(and120_out, p[2], p[1], p[0], cin);
        and and121(and121_out, p[2], p[1], g[0]);
        and and122(and122_out, p[2], g[1]);
        
        and and130(and130_out, p[3], p[2], p[1], p[0], cin);
        and and131(and131_out, p[3], p[2], p[1], g[0]);
        and and132(and132_out, p[3], p[2], g[1]);
        and and133(and133_out, p[3], g[2]);
        
        or  or20(c[0], g[0], and100_out);
        or  or21(c[1], g[1], and110_out, and111_out);
        or  or22(c[2], g[2], and120_out, and121_out, and122_out);
        or  or23(cout, g[3], and130_out, and131_out, and132_out, and133_out);
        
        xor xor30(sum[0], a[0], b[0], cin);
        xor xor31(sum[1], a[1], b[1], c[0]);
        xor xor32(sum[2], a[2], b[2], c[1]);
        xor xor33(sum[3], a[3], b[3], c[2]);
        
        endmodule
    \end{lstlisting}
\end{solution*}
\begin{example}
    What is the total cell area?
\end{example}
\begin{solution*}
    $35.666801$.
\end{solution*}
\begin{example}
    What is the total power consumption?
\end{example}
\begin{solution*}
    $16553~\mathrm{nW}$.
\end{solution*}

Setup a timing constraint 0.28 ns from operand \texttt{a} to summation \texttt{sum} and synthesis again.

\begin{example}
    What is the total cell area?
\end{example}
\begin{solution*}
    79.780998
\end{solution*}
\begin{example}
    What is the total power consumption?
\end{example}
\begin{solution*}
    $27998~\mathrm{nW}$.
\end{solution*}
\begin{example}
    What is the slack?
\end{example}
\begin{solution*}
    $0.05$.
\end{solution*}
\begin{example}
    Attach all scripts you used in file \texttt{cla4.tcl}.
\end{example}
\begin{solution*}
    \lstset{language=tcl,
        basicstyle=\ttfamily\scriptsize
    }
    \begin{lstlisting}
        # Compile Script for Synopsys
        # dc_shell-t -f cla4.tcl
        set top cla4
        set target_library "gscl45nm.db"
        set link_library "gscl45nm.db"
        
        analyze -f verilog ${top}.v
        elaborate ${top}
        compile_ultra
        set filename [format "%s%s"  ${top} ".vh"]
        write_file -f verilog -output $filename
        redirect timing.rep { report_timing }
        redirect cell.rep { report_cell }
        redirect power.rep { report_power }
        
        set_max_delay -from {a*} -to {s*} 0.28
        compile_ultra
        set filename [format "%s%s"  ${top} "-fast.vh"]
        write_file -f verilog -output $filename
        redirect "timing-fast.rep" { report_timing }
        redirect "cell-fast.rep" { report_cell }
        redirect "power-fast.rep" { report_power }
        
        quit
    \end{lstlisting}
\end{solution*}

\end{document}
