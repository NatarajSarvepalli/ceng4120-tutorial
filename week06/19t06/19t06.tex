%!TEX program = xelatex
\documentclass[10pt, compress, handout]{beamer}
\usepackage[titleprogressbar]{../../cls/beamerthemem}

\setbeamertemplate{caption}[numbered]
\setbeamertemplate{theorems}[numbered]
\newcounter{example}
\resetcounteronoverlays{example}
\newtheorem{crl}{Corollary}[theorem]
\newtheorem{eg}[example]{Example}
\newtheorem*{solution*}{Solution}

\usepackage{booktabs}
\usepackage[scale=2]{ccicons}
\usepackage{minted}

\usepackage{cleveref}
\crefname{example}{Example}{Examples}

\usepgfplotslibrary{dateplot}

\usemintedstyle{trac}

\usepackage{algorithm}
\usepackage[noend]{algpseudocode}
\resetcounteronoverlays{algorithm}

\usepackage{listings}
\usepackage{mathtools}
\usepackage{multicol}
\usepackage{qtree}

\usepackage{tikz}

\usepackage{version}
% \excludeversion{proof}
% \excludeversion{solution*}

\makeatletter
\def\old@comma{,}
\catcode`\,=13
\def,{%
    \ifmmode%
    \old@comma\discretionary{}{}{}%
    \else%
    \old@comma%
    \fi%
}
\makeatother

\title{CENG 4120 Tutorial of Week 06}
\subtitle{Unate Recursive Paradigm}
\author{LI Haocheng}
\institute{Department of Computer Science and Engineering}

\begin{document}

\maketitle

\begin{frame}
\frametitle{Tautology}
\begin{eg}
    Consider the function $f = ab + ac + a'$. We question if it is
    a tautology.
\end{eg}
\onslide<2>\begin{solution*}
    The cover is $\begin{matrix}
    01 & 01 & 11\\
    01 & 11 & 01\\
    10 & 11 & 11
    \end{matrix}$.  Assume that
    the first variable is chosen for splitting. We take then the cofactor again w.r.t. $10~11~11$
    and w.r.t. $01~11~11$. The first cofactor has $11~11~11$ in the first row and yields a tautology.
    The second cofactor is $\begin{matrix}
    11 & 01 & 11\\
    11 & 11 & 01
    \end{matrix}$. The cover is unate and there is not a row of $1$s. Hence it is not a tautology and
    $f$ is not a tautology. 
\end{solution*}
\end{frame}

\begin{frame}[fragile]
\frametitle{Containment}
\begin{columns}
    \column{.7\linewidth}
    \begin{theorem}
        \begin{equation}
            f \supseteq \alpha \Leftrightarrow f_\alpha = 1
        \end{equation}
    \end{theorem}
    \onslide<2->\begin{proof}
        \begin{equation}
            f \supseteq \alpha \Leftrightarrow f\alpha = \alpha.
        \end{equation}
        \begin{description}
            \item[$\Rightarrow$] $f_\alpha = f_\alpha \cdot 1 = f_\alpha \alpha_\alpha = (f\alpha)_\alpha = \alpha_\alpha = 1$
            \item[$\Leftarrow$] $f\alpha = (\alpha f_\alpha + \alpha'f_\alpha')\alpha = f_\alpha \alpha = \alpha$
        \end{description}
    \end{proof}

\column{.4\linewidth}
\begin{eg}
    Consider the function $f = ab + ac + a'$. Test whether $f \supseteq bc$.
\end{eg}
\onslide<3>\begin{solution*}
    By taking the cofactor, we obtain $\begin{matrix}
    01 & 11 & 11\\
    01 & 11 & 11\\
    10 & 11 & 11
    \end{matrix}$, which is a tautology. Hence $f \supseteq bc$.
\end{solution*}

\end{columns}

\end{frame}

\begin{frame}[fragile]
\frametitle{Complementation}
\begin{columns}
    \column{.4\linewidth}
    \begin{theorem}
        \begin{equation}
            f' = xf'_x + x'f'_{x'}.
        \end{equation}
    \end{theorem}
    \onslide<2->\begin{proof}
        \begin{align}
            \begin{aligned}
            f' = & (xf_x + x'f_{x'})'\\
            = & (x' + f'_x) (x + f'_{x'})\\
            = & xx' + xf'_x + x'f'_{x'} + f'_x f'_{x'}\\
            = & xf'_x + x'f'_{x'} + (x + x') f'_x f'_{x'}\\
            = & xf'_x + xf'_x f'_{x'} + x'f'_{x'} + x'f'_x f'_{x'}\\
            = & xf'_x (1 + f'_{x'}) + x'f'_{x'} (1 + f'_x)\\
            = & xf'_x + x'f'_{x'}.
            \end{aligned}
        \end{align}
    \end{proof}
    
    \column{.7\linewidth}
    \begin{eg}
        What are the termination rules?
    \end{eg}
    \onslide<3>\begin{solution*}
        \begin{itemize}
            \item $f = 0$. Its complement is the universal cube.
            \item $f = 1$. Its complement is void.
            \item The cover $F$ consists of one implicant. Its complement is computed by
            De Morgan's law.
            \item The cover has a column of 0s, say in position $j$. Let $\alpha$ be a cube of all 1s, except
            in position $j$. Then $f = \alpha f_\alpha$, and $f' = a' + f'_\alpha$. Hence recursively compute
            the complement of $f_\alpha$, and of a and return their union. 
        \end{itemize}
    \end{solution*}
\end{columns}

\end{frame}

\begin{frame}[fragile]
\frametitle{Split Choice}
\begin{columns}
    \column{.9\linewidth}
    \begin{theorem}
        Let $f$ be a posilive unate function in $x$. Then $f' = f'_x + x'f'_{x'}$.
        
        Let $f$ be a negative unate function in $x$. Then $f' = xf'_x + f'_{x'}$.
    \end{theorem}
    \onslide<2>\begin{proof}
        Let $f$ be a posilive unate function in $x$. Then \begin{align}
        \begin{aligned}
        f' = & (xf_x + x'f_{x'})' = (x(f_x + f_{x'}) + x'f_{x'})'\\
        = & (xf_x + xf_{x'} + x'f_{x'})' = (xf_x + (x + x')f_{x'})'\\
        = & (xf_x + f_{x'})' = (x' + f'_x) f'_{x'} = x'f'_{x'} + f'_x f'_{x'}\\
        = & x'f'_{x'} + (f_x + f_{x'})'\\
        = & f'_x + x'f'_{x'}.
        \end{aligned}
        \end{align}
        For negative unate functions the proof is obtained by exchanging $x$ with $x'$. 
    \end{proof}

\end{columns}

\end{frame}

\begin{frame}[fragile]
\frametitle{Complementation Example}
\begin{columns}
    \column{1\linewidth}
    \begin{eg}
        Consider the function $f = ab + ac + a'$. Give the complement.
    \end{eg}
    \onslide<2>\begin{solution*}
        Assume that
        the first variable is chosen for splitting. We take then the cofactor again w.r.t. $10~11~11$
        and w.r.t. $01~11~11$. The first cofactor has $11~11~11$ in the first row and yields void complement.
        
        The second cofactor is $\begin{matrix}
        11 & 01 & 11\\
        11 & 11 & 01
        \end{matrix}$. The cover is unate. Let us select the second
        variable, i.e., $b$. We compute now $f'_a = f'_{ab} + b'f'_{ab'}$. Now $f_{ab}$ is a tautology
        and its complement is void. Then $f_{ab'} = c$ and its complement is $c'$,
        which yields $f'_a = b'c'$. The complement of $f$ is
        $f' = a f'_a = ab'c'$.
    \end{solution*}
    
\end{columns}

\end{frame}

\plain{Questions?}

\end{document}
