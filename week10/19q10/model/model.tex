\documentclass[acmlarge,screen=true,anonymous=false,11pt]{acmart}

% ======================================================== 
%          Copyright & Conference setting
% ======================================================== 
% copyright
\setcopyright{none}
\settopmatter{printacmref=false, printccs=false, printfolios=true}

%\editor{Jennifer B. Sartor}
%\editor{Theo D'Hondt}
%\editor{Wolfgang De Meuter}

\usepackage{afterpage}
\usepackage{pdflscape}
\usepackage{pdfpages}

\usepackage{cleveref}
\crefname{mytheorem}{Theorem}{Theorems}

\usepackage{enumerate}
\usepackage[linguistics]{forest}

\usepackage{pgfplots}
\usepackage{textgreek}
\usepgfplotslibrary{groupplots}
\usetikzlibrary{patterns}

\usepackage{soul}

\usepackage{tikz}
\usetikzlibrary {positioning}
\usetikzlibrary{shapes,arrows}

\newtheorem*{solution*}{Solution}

% ==== defined colors
\definecolor{myorange}{RGB}{244,106,18} %F47012
\definecolor{myblue}{RGB}{0,111,190}    %006FBE
\definecolor{mygreen}{RGB}{0,127,128}   %007F80
\definecolor{myred}{RGB}{228,46,36}     %E42E24
\definecolor{myyellow}{RGB}{198,148,34} %C69422
\definecolor{mydark}{RGB}{114,44,114}   %722C72
\definecolor{mymiddle}{RGB}{144,44,144} %902C90
\definecolor{mylight}{RGB}{167,44,167}  %A72CA7

\definecolor{light-cayenne}{RGB}{177,0,28} %B1001C
\definecolor{light-mocha}{RGB}{179,68,0}   %B34400
\definecolor{ocean}{RGB}{0,55,118}         %003776

\iffalse
\newtoggle{screen}
\toggletrue{screen}
\AtEndPreamble{%
\iftoggle{screen}
{
    \hypersetup{colorlinks, % 01/2018 Bei: change link colors all to blue
        linkcolor=blue,
        citecolor=blue,
        urlcolor=blue,
        filecolor=blue}
}
{
    \hypersetup{hidelinks}
}}
\fi

\begin{document}
\title{CENG 4120 Assignment 2 Sample Solution}
\subtitle{Computer-Aided Design for Very Large Scale Integrated Circuits \\ Due Date: March 04 (Monday) 5:00pm}

\maketitle

\section{Shannon Expansion}
\begin{enumerate}[(i)]
    \item \begin{proof}
        We can divide the SOP terms of $f$ in three categories: $f = xf_+ + \overline{x}f_- + f_0$ where $xf_+$ equals all terms with $x$, $\overline{x}f_-$ equals all terms with $\overline{x}$ and $f_0$ equals all terms without $x$ or $\overline{x}$ then we have $f_{x = 1} = f_+ + f_0$ and $f_{x = 0} = f_- + f_0$.
        Therefore, \begin{align}
            \begin{aligned}
            \overline{f_x} = & \overline{xf_{x = 1} + \overline{x}f_{x = 0}}\\
            = & \overline{x (f_+ + f_0) + \overline{x}(f_- + f_0)}\\
            = & (\overline{x} + \overline{(f_+ + f_0)})(x + \overline{(f_- + f_0)})\\
            = & x\overline{(f_+ + f_0)} + \overline{x}\overline{(f_- + f_0)}\\
            = & x\overline{f}_{x = 1} + \overline{x}\overline{f}_{x = 0} = \left(\overline{f}\right)_x.
            \end{aligned}
        \end{align}.
    \end{proof}
    \item \begin{proof}
        Let $f = xf_+ + \overline{x}f_- + f_0$ and $g = xg_+ + \overline{x}g_- + g_0$, where any of $f_+, f_-, f_0, g_+, g_-, g_0$ is not supported by $x$, then $f + g = x(f_+ + g_+) + \overline{x}(f_- + g_-) + f_0 + g_0$.
        \begin{align}
            \begin{aligned}
            f_x + g_x = & xf_{x = 1} + \overline{x}f_{x = 0} + xg_{x = 1} + \overline{x}g_{x = 0}\\
            = & x(f_+ + f_0) + \overline{x}(f_- + f_0)+ x(g_+ + g_0) + \overline{x}(g_- + g_0)\\
            = & x(f_+ + g_+ + f_0 + g_0) + \overline{x}(f_- + g_- + f_0 + g_0)\\
            = & x(f + g)_{x = 1} + \overline{x}(f + g)_{x = 0}\\
            = & \left(f + g\right)_x.
            \end{aligned}
        \end{align}
    \end{proof}
\end{enumerate}

\section{Unate Recursive Paradigm}
\begin{enumerate}[(i)]
    \item \begin{solution*}
        \begin{figure}
            \begin{forest}
                [$\begin{matrix}
                10 & 01 & 01 & 11\\
                10 & 11 & 10 & 01\\
                01 & 01 & 01 & 01
                \end{matrix}$[$\begin{matrix}
                11 & 01 & 01 & 01
                \end{matrix}$][$\begin{matrix}
                11 & 01 & 01 & 11\\
                11 & 11 & 10 & 01
                \end{matrix}$[$\begin{matrix}
                11 & 01 & 11 & 11
                \end{matrix}$][$\begin{matrix}
                11 & 11 & 11 & 01
                \end{matrix}$]]]
            \end{forest}
        \caption{Recursion Tree}
        \label{fg:2}
        \end{figure}

By the recursion tree in~\Cref{fg:2}, \begin{equation}
    \overline{F(x, y, z, w)} = x (\overline{y} + \overline{z} + \overline{w}) + \overline{x} (\overline{y}z + \overline{z}\overline{w}) = x\overline{y} + x\overline{z} + x\overline{w} + \overline{x}\overline{y}z + \overline{x}\overline{z}\overline{w}.
\end{equation}
    \end{solution*}

\item \begin{solution*}
    \begin{table}
        \caption{Karnaugh Map}
        \label{tb:2}
        \begin{tabular}{c|cccc}
            $zw$\textbackslash$xy$ & $00$ & $01$ & $11$ & $10$ \\\midrule
            $00$ & $0$ & $0$ & $0$ & $0$ \\
            $01$ & $1$ & $1$ & $0$ & $0$ \\
            $11$ & $0$ & $1$ & $1$ & $0$ \\
            $10$ & $0$ & $1$ & $0$ & $0$
        \end{tabular}
    \end{table}
\end{solution*}

The complement function covers all zeros of the Karnaugh map shown in~\Cref{tb:2} but no ones.
\end{enumerate}

\section{Simple Comparator}
\begin{enumerate}[(i)]
    \item \begin{solution*}
        \begin{table}
            \caption{Comparator Karnaugh Map}
            \label{tb:3}
            \begin{tabular}{c|cccc}
                $b_1 b_0$\textbackslash$a_1 a_0$ & $00$ & $01$ & $11$ & $10$ \\\midrule
                $00$ & $1$ & $1$ & $1$ & $1$\\
                $01$ & $0$ & $1$ & $1$ & $1$\\
                $11$ & $0$ & $0$ & $1$ & $0$\\
                $10$ & $0$ & $0$ & $1$ & $1$
            \end{tabular}
        \end{table}

From the Karnaugh map shown in~\Cref{tb:3}, we have \begin{equation}
    z = a_1 \cdot a_0 + a_1 \cdot \overline{b_1} + a_1 \cdot \overline{b_0} + a_0 \cdot \overline{b_1} + \overline{b_1} \cdot \overline{b_0}
\end{equation}
    \end{solution*}

\item \begin{solution*}
    The ROBDD for good order $a_1, b_1, a_0, b_0$ is shown in~\Cref{fg:3:2}.
\begin{figure}
    \newdimen\nodeDist
    \nodeDist=.7cm
    \newdimen\nodeSize
    \nodeSize=.7cm
    \begin{tikzpicture}[
    node2/.style ={rectangle, draw=black, fill=white,
        text width=0.75cm, text centered,
        minimum size=\nodeSize,
        },
    node1/.style={
        draw,
        circle,
        inner sep=0,
        outer sep=0,
        minimum size=\nodeSize,
        node distance=\nodeDist
    },
    ]
    
    \node[node1] (a_1) {$a_1$};
    \node[node1] (b_1_l) [below left = \nodeDist of a_1] {$b_1$};
    \node[node1] (b_1_r) [below right = \nodeDist of a_1] {$b_1$};
    \node[node1] (a_0) [below right = \nodeDist of b_1_l]{$a_0$};
    \node[node1] (b_0) [below = \nodeDist of a_0] {$b_0$};
    \node[node2] (s0) [below left = \nodeDist of b_0] {$0$};
    \node[node2] (s1) [below right = \nodeDist of b_0] {$1$};
    
    \path (a_1) edge[dashed] (b_1_l);
    \path (a_1) edge (b_1_r);
    \path (b_1_l) edge (s0);
    \path (b_1_l) edge[dashed] (a_0);
    \path (b_1_r) edge[dashed] (s1);
    \path (b_1_r) edge (a_0);
    \path (a_0) edge[dashed] (b_0);
    \path (a_0) edge (s1);
    \path (b_0) edge (s0);
    \path (b_0) edge[dashed] (s1);
    \end{tikzpicture}
    \caption{Good Order}
    \label{fg:3:2}
\end{figure}
\end{solution*}

\item \begin{solution*}
    The ROBDD for bad order $a_1, a_0, b_1, b_0$ is shown in~\Cref{fg:3:3}.
\begin{figure}
    \newdimen\nodeDist
    \nodeDist=.7cm
    \newdimen\nodeSize
    \nodeSize=.7cm
    \begin{tikzpicture}[
    node2/.style ={rectangle, draw=black, fill=white,
        text width=0.75cm, text centered,
        minimum size=\nodeSize
    },
    node1/.style={
        draw,
        circle,
        inner sep=0,
        outer sep=0,
        minimum size=\nodeSize,
        node distance=\nodeDist
    },
    ]
    
    \node[node1] (a_1) {$a_1$};
    \node[node1] (a_0_l) [below left = \nodeDist of a_1] {$a_0$};
    \node[node1] (a_0_r) [below right = \nodeDist of a_1] {$a_0$};
    \node[node1] (b_1_l) [below left = \nodeDist of a_0_l] {$b_1$};
    \node[node1] (b_1_m) [below right = \nodeDist of a_0_l] {$b_1$};
    \node[node1] (b_1_r) [below right = \nodeDist of a_0_r] {$b_1$};
    \node[node1] (b_0) [below = \nodeDist of b_1_m] {$b_0$};
    \node[node2] (s0) [below left = \nodeDist of b_0] {$0$};
    \node[node2] (s1) [below right = \nodeDist of b_0] {$1$};
    
    \path (a_1) edge[dashed] (a_0_l);
    \path (a_1) edge (a_0_r);
    \path (a_0_l) edge[dashed] (b_1_l);
    \path (a_0_l) edge (b_1_m);
    \path (a_0_r) edge[dashed] (b_1_r);
    \path (a_0_r) edge (s1);
    \path (b_1_l) edge (s0);
    \path (b_1_l) edge[dashed] (b_0);
    \path (b_1_m) edge[dashed] (s1);
    \path (b_1_m) edge (s0);
    \path (b_1_r) edge[dashed] (s1);
    \path (b_1_r) edge (b_0);
    \path (b_0) edge[dashed] (s1);
    \path (b_0) edge (s0);
    \end{tikzpicture}
    \caption{Good Order}
    \label{fg:3:3}
\end{figure}
\end{solution*}
\end{enumerate}

\section{ITE Recursion}
\begin{solution*}
    The top-level ITE tuple is \begin{equation}
        f = a \oplus b = ITE(a, \overline{b}, b).
    \end{equation}
    Let the topmost variable be $a$ and label the topmost node accordingly, then the lower diagram is \begin{equation}
        f_l = ITE(0, \overline{b}, b) = b,
    \end{equation} and the upper diagram is \begin{equation}
        f_h = ITE(1, \overline{b}, b) = \overline{b}.
    \end{equation}
    The BDD in~\Cref{fg:4} is achieved after simplification.
    \begin{figure}
        \newdimen\nodeDist
        \nodeDist=.7cm
        \newdimen\nodeSize
        \nodeSize=1cm
        \begin{tikzpicture}[
        node2/.style ={rectangle, draw=black, fill=white,
            text width=0.7cm, text centered,
            minimum height=1cm},
        node1/.style={
            draw,
            circle,
            inner sep=0,
            outer sep=0,
            minimum size=\nodeSize,
            node distance=\nodeDist
        },
        ]
        
        \node[node1] (a) {$a$};
        \node[node1] (b_l) [below left = \nodeDist of a] {$b$};
        \node[node1] (b_r) [below right = \nodeDist of a] {$b$};
        \node[node2] (s0) [below = \nodeDist of b_l] {$0$};
        \node[node2] (s1) [below = \nodeDist of b_r] {$1$};

        \path (a) edge[dashed] (b_l);
        \path (a) edge (b_r);
        \path (b_l) edge (s1);
        \path (b_l) edge[dashed] (s0);
        \path (b_r) edge[dashed] (s1);
        \path (b_r) edge (s0);
        \end{tikzpicture}
        \caption{BDD}
        \label{fg:4}
    \end{figure}
\end{solution*}

\section{Positional Cube Notation}
\begin{enumerate}[(i)]
    \item \begin{solution*}
        $F = a \cdot b \cdot c + a \cdot b \cdot \overline{c} + \overline{a} \cdot b + \overline{b} \cdot \overline{c} \cdot \overline{d} = \begin{matrix}
        01 & 01 & 01 & 11\\
        01 & 01 & 10 & 11\\
        10 & 01 & 11 & 11\\
        11 & 10 & 10 & 10
        \end{matrix}.$
    \end{solution*}

\item \begin{solution*}
    $\overline{F} = \overline{b}c + \overline{b}d = \begin{matrix}
    11 & 10 & 01 & 11\\
    11 & 10 & 11 & 01
    \end{matrix}$.
\end{solution*}

\item \begin{solution*}
    The blocking matrix of $a \cdot b \cdot c$ is \begin{tabular}{c|cc}
        & $\overline{b}c$ & $\overline{b}d$\\\midrule
        $a$ & &\\
        $b$ & $1$ & $1$\\
        $c$ & &
    \end{tabular}. Thus, it can be expanded as $b$.

The blocking matrix of $a \cdot b \cdot \overline{c}$ is \begin{tabular}{c|cc}
    & $\overline{b}c$ & $\overline{b}d$\\\midrule
    $a$ & &\\
    $b$ & $1$ & $1$\\
    $\overline{c}$ & $1$ &
\end{tabular}. Thus, it can be expanded as $b$.

The blocking matrix of $\overline{a} \cdot b$ is \begin{tabular}{c|cc}
    & $\overline{b}c$ & $\overline{b}d$\\\midrule
    $\overline{a}$ & &\\
    $b$ & $1$ & $1$
\end{tabular}. Thus, it can be expanded as $b$.

The blocking matrix of $\overline{b} \cdot \overline{c} \cdot \overline{d}$ is \begin{tabular}{c|cc}
    & $\overline{b}c$ & $\overline{b}d$\\\midrule
    $\overline{b}$ & &\\
    $\overline{c}$ & $1$ &\\
    $\overline{d}$ & & $1$
\end{tabular}. Thus, it can be expanded as $\overline{c} \cdot \overline{d}$.
\end{solution*}

\item $F = b + \overline{c} \cdot \overline{d} = \begin{matrix}
11 & 01 & 11 & 11\\
11 & 11 & 10 & 10
\end{matrix}$.
\end{enumerate}

\section{Kernel}
\begin{enumerate}[(i)]
    \item \begin{solution*}
        \hspace{-2in}
        \resizebox{1.4\linewidth}{!}{
        \begin{forest}
            [$abcd + abf + abh + pqcw + pqh + cdefw$[$\begin{matrix}
            \mathrm{cubes} & abcd + abf + abh\\
            \mathrm{co} & ab\\
            \mathrm{kernel} & cd + f + h
            \end{matrix}$[no work]][$\begin{matrix}
            abcd + abf + abh\\
            ab\\
            cd + f + h
            \end{matrix}$[no work]][$\begin{matrix}
            abcd + pqcw + cdefw\\
            c\\
            abd + pqw + defw
            \end{matrix}$[$\begin{matrix}
            abd + defw\\
            d\\
            ab + efw
            \end{matrix}$[no work]][$\begin{matrix}
            pqw + defw\\
            w\\
            pq + def
            \end{matrix}$[no work]]][$\begin{matrix}
            abcd + cdefw\\
            cd\\
            ab + efw
            \end{matrix}$[no work]][$\begin{matrix}
            abf + cdefw\\
            f\\
            ab + cdew
            \end{matrix}$[no work]][$\begin{matrix}
            abh + pqh\\
            h\\
            ab + pq
            \end{matrix}$[no work]][$\begin{matrix}
            pqcw + pqh\\
            pq\\
            cw + h
            \end{matrix}$[no work]][$\begin{matrix}
            pqcw + pqh\\
            pq\\
            cw + h
            \end{matrix}$[no work]][$\begin{matrix}
            pqcw + cdefw\\
            cw\\
            pq + def
            \end{matrix}$[no work]]]
        \end{forest}}
    
    Therefore, the corresponding kernels and co-kernels of $F$ are \begin{tabular}{cc}
        kernel & co-kernel \\\midrule
        $cd + f + h$ & $ab$\\
        $ab + efw$ & $cd$\\
        $pq + def$ & $cw$\\
        $abd + pqw + defw$ & $c$\\
        $ab + cdew$ & $f$\\
        $ab + pq$ & $h$\\
        $cw + h$ & $pq$\\
        $abcd + abf + abh + pqcw + pqh + cdefw$ & $1$
    \end{tabular}.
    
    \begin{forest}
        [$abce + abg + pqcw + pqg$[$\begin{matrix}
        \mathrm{cubes} & abce + abg\\
        \mathrm{co} & ab\\
        \mathrm{kernel} & ce + g
        \end{matrix}$[no work]][$\begin{matrix}
        abce + abg\\
        ab\\
        ce + g
        \end{matrix}$[no work]][$\begin{matrix}
        abce + pqcw\\
        c\\
        abe + pqw
        \end{matrix}$[no work]][$\begin{matrix}
        abg + pqg\\
        g\\
        ab + pq
        \end{matrix}$[no work]][$\begin{matrix}
        pqcw + pqg\\
        pq\\
        cw + g
        \end{matrix}$[no work]][$\begin{matrix}
        pqcw + pqg\\
        pq\\
        cw + g
        \end{matrix}$[no work]]]
    \end{forest}

Therefore, the corresponding kernels and co-kernels of $G$ are \begin{tabular}{cc}
    kernel & co-kernel \\\midrule
    $ce + g$ & $ab$\\
    $abe + pqw$ & $c$\\
    $ab + pq$ & $g$\\
    $cw + g$ & $pq$\\
    $abce + abg + pqcw + pqg$ & $1$
\end{tabular}.

    \end{solution*}

\item \begin{solution*}
    \begin{tabular}{c|ccccccccccc}
    & a & b & c & d & e & f & g & h & p & q & w \\\midrule
    abcd & \hl{1} & \hl{1} & 1 & 1 &   &   &   &   &   &   &   \\
    abf  & \hl{1} & \hl{1} &   &   &   & 1 &   &   &   &   &   \\
    abh  & \hl{1} & \hl{1} &  &  &   &   &   & 1 &   &   &   \\
    pqcw &   &   & 1 &   &   &   &   &   & 1 & 1 & 1 \\
    pqh  &   &   &   &   &   &   &   & 1 & 1 & 1 &   \\
    cdefw&   &   & 1 & 1 & 1 & 1 &   &   &   &   & 1 \\\midrule
    abce  & \hl{1} & \hl{1} & 1 &   & 1 &   &   &   &   &   &   \\
    abg   & \hl{1} & \hl{1} &   &   &   &   & 1 &   &   &   &   \\
    pqcw  &   &   & 1 &   &   &   &   &   & 1 & 1 & 1 \\
    pqg   &   &   &   &   &   &   & 1 &   & 1 & 1 &
\end{tabular}.

The single cube divisor is $ab$. The number of saved literals is $(2 - 1) \times 5 - 2 = 3$.
\end{solution*}

\item \begin{solution*}
    \begin{tabular}{c|cccccccccccccccccccccccc}
        & cd & f & h & abd & pqw & defw & efw & ab & cdew & def & ce & g & abe & cw & pq  \\\midrule
        ab  &  1 & 1 & 1 &     &     &      &     &    &      &     &    &   &     &    &    \\
        c   &    &   &   &  1  &  1  &   1  &     &    &      &     &    &   &     &    &    \\
        cd  &    &   &   &     &     &      &  1  &  1 &      &     &    &   &     &    &    \\
        f   &    &   &   &     &     &      &     &  1 &   1  &     &    &   &     &    &    \\
        h   &    &   &   &     &     &      &     &  \hl{1} &      &     &    &   &     &    &  \hl{1}  \\
        pq  &    &   & 1 &     &     &      &     &    &      &     &    &   &     &  1 &    \\
        cw  &    &   &   &     &     &      &     &    &      &   1 &    &   &     &    &  1 \\\midrule
        ab  &    &   &   &     &     &      &     &    &      &     & 1  &  1&     &    &    \\
        c   &    &   &   &     &  1  &      &     &    &      &     &    &   &  1  &    &    \\
        pq  &    &   &   &     &     &      &     &    &      &     &    & 1 &     &  1 &    \\
        g   &    &   &   &     &     &      &     &  \hl{1} &      &     &    &   &     &    & \hl{1}
    \end{tabular}.

The multiple cube divisor found is $ab+pq$. The number of saved literals is $12 - 4 - 4 = 4$.
\end{solution*}
\end{enumerate}

\section{Shannon Expansion}
\begin{enumerate}[(i)]
    \item \begin{solution*}
        $F = (x + F_{\overline{x}})(\overline{x} + F_x)$.
    \end{solution*}

\item \begin{proof}
    \begin{align}
        \begin{aligned}
        (x + F_{\overline{x}})(\overline{x} + F_x) = & xF_x + \overline{x}F_{\overline{x}} + F_x F_{\overline{x}}\\
        = & xF_x + \overline{x}F_{\overline{x}} + (x + \overline{x}) F_x F_{\overline{x}}\\
        = & xF_x + x F_x F_{\overline{x}} + \overline{x}F_{\overline{x}} + \overline{x} F_x F_{\overline{x}}\\
        = & xF_x(1 + F_{\overline{x}}) + \overline{x}F_{\overline{x}}(1 + F_x)\\
        = & xF_x + \overline{x}F_{\overline{x}}.
        \end{aligned}
    \end{align}
\end{proof}
\end{enumerate}

\end{document}
